\documentclass[12pt,a4paper]{article}

\usepackage{amsthm}
\usepackage{amsmath}
\usepackage{amsfonts}
\usepackage{tikz}
\usepackage{wrapfig}
\usepackage{multicol}
\usepackage{color}
\usepackage{forest}

\usepackage[utf8]{inputenc}
\usepackage[english,russian]{babel}
\usepackage{enumitem}
\usepackage{multirow}

\usetikzlibrary{positioning, shapes}
%-------------------------------------------

\setlength{\textwidth}{7.0in}
\setlength{\oddsidemargin}{-0.35in}
\setlength{\topmargin}{-0.5in}
\setlength{\textheight}{9.0in}
\setlength{\parindent}{0.3in}

\newtheorem{theorem}{Theorem}
\newtheorem{task}[theorem]{Задача}
\addto\captionsrussian{\renewcommand*{\proofname}{Решение}}

\DeclareMathOperator{\LB}{LB}
\DeclareMathOperator{\UB}{UB}

\begin{document}
\begin{flushright}
\textbf{
  ЛЭТИ, гр. 8303, Гришин Константин \\
  \today
}
\end{flushright}

\begin{center}
\textbf{Машинное обучение} \\
\textbf{Практические задания №5}             \\
\end{center}

\begin{task}
  Дан набор значений $[2, 4, 10, 12, 3, 20, 30, 11, 25]$. Предположим, что
  количество кластеров $k=3$, и выбраны начальные средние значения $m_1=2$,
  $m_2=4$, $m_3=6$. Покажите, какие кластеры будут сформированы после первой
  итерации алгоритма k-средних, и рассчитайте новые значения центров кластеров
  для следующей итерации.
\end{task}

\begin{proof}
  Рассчитаем расстояние каждой точки данных до начальных значений.

  \begin{center}
    \begin{tabular}{c|c|c|c|c}
         & 2  & 4  & 6  & cluster \\
      \hline
      2  & 0  & 2  & 4  & 2 \\
      4  & 2  & 0  & 2  & 4 \\
      10 & 8  & 6  & 4  & 6 \\ 
      12 & 10 & 8  & 6  & 6 \\ 
      3  & 1  & 1  & 3  & 2 \\
      20 & 18 & 16 & 14 & 6 \\ 
      30 & 28 & 26 & 24 & 6 \\ 
      11 & 9  & 7  & 5  & 6 \\ 
      25 & 23 & 21 & 19 & 6
    \end{tabular}
  \end{center}

  Получены кластеры данных, проведем расчет новых центров.

  \begin{center}
    \begin{tabular}{|c|l|c|}
      \hline
      Cluster & Data & Mean \\
      \hline
      $C_1$ & 2, 3                   & 2.5 \\
      $C_2$ & 4                      & 4   \\
      $C_3$ & 10, 12, 20, 30, 11, 25 & 18  \\ 
      \hline
    \end{tabular}
  \end{center}
\end{proof}

\begin{task}
  Дан набор точек $x$ и вероятности их принадлежности к кластерам $C_1$ и $C_2$

  \begin{center}
    \begin{tabular}{|c|c|c|}
      \hline
      $x$ & $P(C_1|x)$ & $P(C_2|x)$ \\
      \hline
      2 & 0.9 & 0.1 \\
      3 & 0.8 & 0.1 \\
      7 & 0.3 & 0.7 \\
      9 & 0.1 & 0.9 \\
      2 & 0.9 & 0.1 \\
      1 & 0.8 & 0.2 \\
      \hline
    \end{tabular}
  \end{center}

  \begin{enumerate}[label=\Alph*., leftmargin=4\parindent]
    \item Найдите оценку максимального правдоподобия для средних $\mu_1$ $\mu_2$.
    \item Предположим, что $\mu_1=2$, $\mu_2=7$ и $\sigma_1=\sigma_2=1$.
          Найдите вероятности принадлежности точки $x=5$ к кластерам $C_1$ и $C_2$.
          Априорные вероятности каждого кластера $P(C_1)=P(C_2)=0.5$ и $P(x=5)=0.029$.
  \end{enumerate}
\end{task}

\begin{proof}\hfill
  \begin{enumerate}[label=\Alph*., leftmargin=4\parindent]
    \item $\mu_i$ - средневзвешенное всех точек:
          $$
          \mu_i=\frac{\sum_{j=1}^{n} w_{ij} x_j}{\sum_{j=1}^{n} w_{ij}}
          $$
        
          \begin{verbatim}
            >>> w1 = np.array([0.9, 0.8, 0.3, 0.1, 0.9, 0.8])
            >>> w2 = np.array([0.1, 0.1, 0.7, 0.9, 0.1, 0.2])
            >>> x = np.array([2, 3, 7, 9, 2, 1])
            >>> (w1 * x).sum() / w1.sum()
            2.5789473684210535
            >>> (w2 * x).sum() / w2.sum()
            6.619047619047618
          \end{verbatim}

          $\mu_1=2.58$, $\mu_2=6.62$

    \item Вероятность нахождения точки в кластере:
          $$
          P(C_i|x_j)=\frac{f_i(x_j) \cdot P(C_i)}
                          {\sum_{a=1}^{k} f_a(x_j) \cdot P(C_a)}
          $$
          $$
          f_i(x)=f(x_j|\mu_i,\sigma_i^2)=\frac{1}{\sqrt{2\pi}\sigma_i}
          exp(-\frac{(x-\mu_i)^2}{2\sigma_i^2})
          $$

          \begin{verbatim}
            >>> f = lambda x, mean, std:
            np.exp(-(x-mean)**2/(2*std**2))/(np.sqrt(2*np.pi)*std)
            >>> pc1 = f(5, 2, 1) * 0.5
            >>> pc2 = f(5, 7, 1) * 0.5
            >>> pc1 / (pc1+pc2)
            0.07585818002124355
            >>> pc2 / (pc1+pc2)
            0.9241418199787564
          \end{verbatim}

          $P(C_1|5)=0.076$, $P(C_2|5)=0.924$
  \end{enumerate}
\end{proof}
\clearpage

\begin{task}
  Даны категориальные данные размерности:

  \begin{center}
    \begin{tabular}{|c|c c c c c|}
      \hline
      Point & $X_1$ & $X_2$ & $X_3$ & $X_4$ & $X_5$ \\
      \hline
      $x_1$ & 1 & 0 & 1 & 1 & 0 \\
      $x_2$ & 1 & 1 & 0 & 1 & 0 \\
      $x_3$ & 0 & 0 & 1 & 1 & 0 \\
      $x_4$ & 0 & 1 & 0 & 1 & 0 \\
      $x_5$ & 1 & 0 & 1 & 0 & 1 \\
      $x_6$ & 0 & 1 & 1 & 0 & 0 \\
      \hline
    \end{tabular}
  \end{center}

  Близость двух наблюдений определяется через количество совпадений и
  несовпадений значений признаков. Допустим, что $n_{11}$ количество признаков
  одновременно равных 1 для наблюдений $x_i$ и $x_j$, и $n_{10}$ количество
  признаков равных 1 для наблюдения $x_i$ и в то же время равных 0 для
  наблюдений $x_j$. По аналогии определяются значения для $n_{01}$ и $n_{00}$:

  \begin{center}
    \begin{tabular}{|c|c|c|c|}
      \hline
            & \multicolumn{3}{c|}{\(x_j\)} \\
      \hline
            &   & 1 & 0 \\
      \cline{2-4}
        \multirow{2}{*}{\(x_i\)} & 1 & $n_{11}$ & $n_{10}$ \\
                                 & 0 & $n_{01}$ & $n_{00}$ \\
      \hline
    \end{tabular}
  \end{center}

  Определим следующие метрики:
  \begin{itemize}
    \item Коэффициент простого совпадения
          $$SMC(x_i, x_j) = \frac{n_{11}+n_{00}}{n_{11}+n_{10}+n_{01}+n_{00}}$$
    \item Коэффициент Жаккара
          $$JC(x_i, x_j) = \frac{n_{11}}{n_{11}+n_{10}+n_{01}}$$
    \item Коэффициент Рассела и Рао
          $$RC(x_i, x_j) = \frac{n_{11}}{n_{11}+n_{10}+n_{01}+n_{00}}$$
  \end{itemize}

  Постройте дендограммы полученные после иерархической кластеризации при
  следующих параметрах:
  \begin{itemize}
    \item Метод одиночной связи с метрикой RC
    \item Метод полной связи с метрикой SMC
    \item Невзвешенный центройдный метод с метрикой JC
  \end{itemize}
\end{task}
\clearpage

\begin{proof}\hfill
  \begin{itemize}
    \item Составим таблицу расстояний метрики RC
          \begin{center}
            \begin{tabular}{|c|c c c c c|}
                    &$x_1$&$x_2$&$x_3$&$x_4$&$x_5$\\
              \hline
              $x_2$ & 0.4 &     &     &     &     \\
              $x_3$ & 0.4 & 0.2 &     &     &     \\
              $x_4$ & 0.2 & 0.4 & 0.2 &     &     \\
              $x_5$ & 0.4 & 0.2 & 0.2 & 0.0 &     \\
              $x_6$ & 0.2 & 0.2 & 0.2 & 0.2 & 0.2 \\
            \end{tabular}
          \end{center}

          Построим дендограмму методом одиночной связи
          \begin{center}
            \begin{tikzpicture}[
              x=30mm, y=20mm,
              v/.style = {draw, ellipse}
            ]
              \node[v] (1)  at (0, 0) {$x_1$};
              \node[v] (6)  at (1, 0) {$x_6$};
              \node[v] (4)  at (2, 0) {$x_4$};
              \node[v] (5)  at (3, 0) {$x_5$};
              \node[v] (2)  at (4, 0) {$x_2$};
              \node[v] (3)  at (5, 0) {$x_3$};

              \node[v] (16)  at (0.5, 1) {$x_1x_6$};
              \node[v] (45)  at (2.5, 1) {$x_4x_5$};
              \node[v] (23)  at (4.5, 1) {$x_2x_3$};

              \node[v] (1456)  at (1.5, 2) {$x_1x_4x_5x_6$};
              \node[v] (123456)  at (2.5, 4) {$x_1x_2x_3x_4x_5x_6$};
        
              \draw (1) -- (16);
              \draw (2) -- (23);
              \draw (3) -- (23);
              \draw (4) -- (45);
              \draw (5) -- (45);
              \draw (6) -- (16);
              
              \draw (16) -- (1456);
              \draw (45) -- (1456);

              \draw (23) -- (123456);
              \draw (1456) -- (123456);
            \end{tikzpicture}
          \end{center}
    \clearpage
    \item Составим таблицу расстояний метрики SMC:
          \begin{center}
            \begin{tabular}{|c|c c c c c|}
                    &$x_1$&$x_2$&$x_3$&$x_4$&$x_5$\\
              \hline
              $x_2$ & 0.6 &     &     &     &     \\
              $x_3$ & 0.8 & 0.4 &     &     &     \\
              $x_4$ & 0.4 & 0.8 & 0.6 &     &     \\
              $x_5$ & 0.6 & 0.2 & 0.4 & 0.0 &     \\
              $x_6$ & 0.4 & 0.4 & 0.6 & 0.6 & 0.4 \\
            \end{tabular}
          \end{center}

          Построим дендограмму методом полной связи:
          \begin{center}
            \begin{tikzpicture}[
              x=30mm, y=20mm,
              v/.style = {draw, ellipse}
            ]
              \node[v] (1)  at (0, 0) {$x_1$};
              \node[v] (6)  at (1, 0) {$x_6$};
              \node[v] (4)  at (2, 0) {$x_4$};
              \node[v] (5)  at (3, 0) {$x_5$};
              \node[v] (2)  at (4, 0) {$x_2$};
              \node[v] (3)  at (5, 0) {$x_3$};

              \node[v] (16)  at (0.5, 1) {$x_1x_6$};
              \node[v] (45)  at (2.5, 1) {$x_4x_5$};
              \node[v] (23)  at (4.5, 1) {$x_2x_3$};

              \node[v] (1456)  at (1.5, 2) {$x_1x_4x_5x_6$};
              \node[v] (123456)  at (2.5, 4) {$x_1x_2x_3x_4x_5x_6$};
        
              \draw (1) -- (16);
              \draw (2) -- (23);
              \draw (3) -- (23);
              \draw (4) -- (45);
              \draw (5) -- (45);
              \draw (6) -- (16);
              
              \draw (16) -- (1456);
              \draw (45) -- (1456);

              \draw (23) -- (123456);
              \draw (1456) -- (123456);
            \end{tikzpicture}
          \end{center}
          \clearpage

          \item Составим таблицу расстояний метрики JC
                \begin{center}
                  \begin{tabular}{|c|c c c c c|}
                          &$x_1$&$x_2$&$x_3$&$x_4$&$x_5$\\
                    \hline
                    $x_2$ & 0.50 &      &      &      &      \\
                    $x_3$ & 0.67 & 0.25 &      &      &      \\
                    $x_4$ & 0.25 & 0.67 & 0.33 &      &      \\
                    $x_5$ & 0.50 & 0.20 & 0.25 & 0.0  &      \\
                    $x_6$ & 0.25 & 0.25 & 0.33 & 0.33 & 0.25 \\
                  \end{tabular}
                \end{center}

                Построим дендограмму методом полной связи:
                \begin{center}
                  \begin{tikzpicture}[
                    x=30mm, y=20mm,
                    v/.style = {draw, ellipse}
                  ]
                    \node[v] (1)  at (0, 0) {$x_1$};
                    \node[v] (6)  at (1, 0) {$x_6$};
                    \node[v] (4)  at (2, 0) {$x_4$};
                    \node[v] (5)  at (3, 0) {$x_5$};
                    \node[v] (2)  at (4, 0) {$x_2$};
                    \node[v] (3)  at (5, 0) {$x_3$};

                    \node[v] (16)  at (0.5, 1) {$x_1x_6$};
                    \node[v] (45)  at (2.5, 1) {$x_4x_5$};
                    \node[v] (23)  at (4.5, 1) {$x_2x_3$};

                    \node[v] (1456)  at (1.5, 2) {$x_1x_4x_5x_6$};
                    \node[v] (123456)  at (2.5, 4) {$x_1x_2x_3x_4x_5x_6$};
              
                    \draw (1) -- (16);
                    \draw (2) -- (23);
                    \draw (3) -- (23);
                    \draw (4) -- (45);
                    \draw (5) -- (45);
                    \draw (6) -- (16);
                    
                    \draw (16) -- (1456);
                    \draw (45) -- (1456);

                    \draw (23) -- (123456);
                    \draw (1456) -- (123456);
                  \end{tikzpicture}
                \end{center}
  \end{itemize}

  Для указанных комбинаций метрики и метода определения расстояния между
  кластерами, разбиение на классы одинаково. Различаются только итоговые
  расстояния между точками и кластерами.
\end{proof}

\end{document}
