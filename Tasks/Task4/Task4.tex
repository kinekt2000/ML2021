\documentclass[12pt,a4paper]{article}

\usepackage{amsthm}
\usepackage{amsmath}
\usepackage{amsfonts}
\usepackage{tikz}
\usepackage{wrapfig}
\usepackage{multicol}
\usepackage{color}
\usepackage{forest}

\usepackage[utf8]{inputenc}
\usepackage[english,russian]{babel}
\usepackage{enumitem}
\usepackage{multirow}

\usetikzlibrary{positioning, shapes}
%-------------------------------------------

\setlength{\textwidth}{7.0in}
\setlength{\oddsidemargin}{-0.35in}
\setlength{\topmargin}{-0.5in}
\setlength{\textheight}{9.0in}
\setlength{\parindent}{0.3in}

\newtheorem{theorem}{Theorem}
\newtheorem{task}[theorem]{Задача}
\addto\captionsrussian{\renewcommand*{\proofname}{Решение}}

\DeclareMathOperator{\LB}{LB}
\DeclareMathOperator{\UB}{UB}

\begin{document}
\begin{flushright}
\textbf{
  ЛЭТИ, гр. 8303, Гришин Константин \\
  \today
}
\end{flushright}

\begin{center}
\textbf{Машинное обучение} \\
\textbf{Практические задания №4}             \\
\end{center}

\begin{task}
  Дан набор данных

  \begin{center}
    \begin{tabular}{|c|c|c|}
      \hline
      Tid & Itemset \\
      \hline
      1 & ACD \\
      2 & BCD \\
      3 & AC \\
      4 & ABD \\
      5 & ABCD \\
      6 & BCD \\
      \hline
    \end{tabular}
  \end{center}

  Найдите все минимальные генераторы для минимального уровня поддержки = 1.

  Множество является минимальным генератором, когда оно не имеет подмножеств
  с тем же уровнем поддержки.
\end{task}

\begin{proof}
  Построим решетку наборов и их частот.
  \begin{center}
    \begin{tikzpicture}[
      x=30mm, y=20mm,
      plain/.style = {draw=none, fill=none},
      round/.style = {draw, ellipse}
    ]
      \node[plain] (empty) at (2.5, 4) {$\emptyset(6)$};
      \node[plain] (ABCD)  at (2.5, 0) {$ABCD(1)$};
      
      \node[round] (A)     at (1, 3)   {$A(4)$};
      \node[round] (B)     at (2, 3)   {$B(4)$};
      \node[round] (C)     at (3, 3)   {$C(5)$};
      \node[round] (D)     at (4, 3)   {$D(5)$};

      \node[round] (AB)    at (0, 2)   {$AB(2)$};
      \node[round] (AC)    at (1, 2)   {$AC(3)$};
      \node[round] (AD)    at (2, 2)   {$AD(3)$};
      \node[round] (BC)    at (3, 2)   {$BC(3)$};
      \node[plain] (BD)    at (4, 2)   {$BD(4)$};
      \node[round] (CD)    at (5, 2)   {$CD(4)$};

      \node[round] (ABC)   at (1, 1)   {$ABC(1)$};
      \node[plain] (ABD)   at (2, 1)   {$ABD(2)$};
      \node[round] (ACD)   at (3, 1)   {$ACD(2)$};
      \node[plain] (BCD)   at (4, 1)   {$BCD(3)$};

      \draw (empty) -- (A);
      \draw (empty) -- (B);
      \draw (empty) -- (C);
      \draw (empty) -- (D);

      \draw (A) -- (AB); \draw (A) -- (AC); \draw (A) -- (AD);
      \draw (B) -- (AB); \draw (B) -- (BC); \draw (B) -- (BD) [ultra thick];
      \draw (C) -- (AC); \draw (C) -- (BC); \draw (C) -- (CD);
      \draw (D) -- (AD); \draw (D) -- (BD); \draw (D) -- (CD);

      \draw (ABC) -- (AB); \draw (ABC) -- (AC); \draw (ABC) -- (BC);
      \draw (ABD) -- (AD); \draw (ABD) -- (BD); \draw (ABD) -- (AB) [ultra thick];
      \draw (ACD) -- (AC); \draw (ACD) -- (AD); \draw (ACD) -- (CD);
      \draw (BCD) -- (BD); \draw (BCD) -- (CD); \draw (BCD) -- (BC) [ultra thick]; 

      \draw (ABCD) -- (ABC);
      \draw (ABCD) -- (ABD);
      \draw (ABCD) -- (ACD) [ultra thick];
      \draw (ABCD) -- (BCD);
    \end{tikzpicture}
  \end{center}

  Наборы, которые являются минимальными генераторами обведены.
  Жирным выделены ребра, которые указывают на подмножества с такой же поддержкой.
  Множество минимальных генераторов: $\{A, B, C, D, AB, AC, AD, BC, CD, ABC, ACD\}$.
\end{proof}
\clearpage

\begin{task}
  Дана решетка наборов и их частоты.
  \begin{center}
    \begin{tikzpicture}[
      x=30mm, y=20mm,
      vertex/.style = {draw, ellipse},
      filled/.style = {draw, ellipse, fill=black!10}
    ]
      \node[vertex] (empty) at (2.5, 4) {$\emptyset(6)$};
      \node[filled] (ABCD)  at (2.5, 0) {$ABCD(1)$};
      
      \node[filled] (A)     at (1, 3)   {$A(6)$};
      \node[vertex] (B)     at (2, 3)   {$B(5)$};
      \node[vertex] (C)     at (3, 3)   {$C(4)$};
      \node[vertex] (D)     at (4, 3)   {$D(3)$};

      \node[filled] (AB)    at (0, 2)   {$AB(5)$};
      \node[filled] (AC)    at (1, 2)   {$AC(4)$};
      \node[filled] (AD)    at (2, 2)   {$AD(3)$};
      \node[vertex] (BC)    at (3, 2)   {$BC(3)$};
      \node[vertex] (BD)    at (4, 2)   {$BD(2)$};
      \node[vertex] (CD)    at (5, 2)   {$CD(2)$};

      \node[filled] (ABC)   at (1, 1)   {$ABC(3)$};
      \node[filled] (ABD)   at (2, 1)   {$ABD(2)$};
      \node[filled] (ACD)   at (3, 1)   {$ACD(2)$};
      \node[vertex] (BCD)   at (4, 1)   {$BCD(1)$};

      \draw (empty) -- (A) [ultra thick];
      \draw (empty) -- (B);
      \draw (empty) -- (C);
      \draw (empty) -- (D);

      \draw (A) -- (AB); \draw (A) -- (AC); \draw (A) -- (AD);
      \draw (B) -- (AB) [ultra thick]; \draw (B) -- (BC); \draw (B) -- (BD);
      \draw (C) -- (AC) [ultra thick]; \draw (C) -- (BC); \draw (C) -- (CD);
      \draw (D) -- (AD) [ultra thick]; \draw (D) -- (BD); \draw (D) -- (CD);

      \draw (ABC) -- (AB); \draw (ABC) -- (AC); \draw (ABC) -- (BC) [ultra thick];
      \draw (ABD) -- (AD); \draw (ABD) -- (BD) [ultra thick]; \draw (ABD) -- (AB);
      \draw (ACD) -- (AC); \draw (ACD) -- (AD); \draw (ACD) -- (CD) [ultra thick];
      \draw (BCD) -- (BD); \draw (BCD) -- (CD); \draw (BCD) -- (BC); 

      \draw (ABCD) -- (ABC);
      \draw (ABCD) -- (ABD);
      \draw (ABCD) -- (ACD);
      \draw (ABCD) -- (BCD) [ultra thick];
    \end{tikzpicture}
  \end{center}

  Выполните следущие задания:
  \begin{enumerate}[label=\Alph*., leftmargin=4\parindent]
    \item Выпишите список всех замкнутых наборов (closed itemsets).
    Множество замкнутое, когда нет надмножеств с тем же уровнем поддержки.
    \item Является ли набор BCD выводимым? Является ли набор ABCD выводимым?
    Какие границы их поддержки? 
  \end{enumerate}
\end{task}

\begin{proof} \hfill
  \begin{enumerate}[label=\Alph*.]
    \item На графе выделены ребра, которые ведут кнадмножествам с
    такой же поддержкой.\\
    Множество замкнутых наборов:
    $\{A, AB, AC, AD, ABC, ABD, ACD, ABCD\}$.
    \item
    \renewcommand{\arraystretch}{1.5} \begin{tabular}[t]{|c|c|c|}
      \hline
      \multirow{8}{*}{\(sup(BCD)\)}
      & $\ge 0$           & $Y = BCD$ \\\cline{2-3}
      & $\le sup(BC) = 3$ & $Y = BC$ \\\cline{2-3}
      & $\le sup(BD) = 2$ & $Y = BD$ \\\cline{2-3}
      & $\le sup(CD) = 2$ & $Y = CD$ \\\cline{2-3}
      & $\ge sup(BC) + sup(BD) - sup(B) = 0$ & $Y = B$ \\\cline{2-3}
      & $\ge sup(BC) + sup(CD) - sup(C) = 0$ & $Y = C$ \\\cline{2-3}
      & $\ge sup(BD) + sup(CD) - sup(D) = 0$ & $Y = D$ \\\cline{2-3}
      & $\begin{array}{lcl} \le sup(BC) + sup(BD) + sup(CD) - \\
        -sup(B) - sup(C) - sup(D) + sup(\emptyset) = 1\end{array}$ & $Y = \emptyset$ \\
      \hline
    \end{tabular}
    \renewcommand{\arraystretch}{1}

    $\LB(BCD)=\{0, 1\}, \UB(BCD)=\{1, 2, 3\}$

    $\max(\LB(BCD)) = \min(\UB(BCD)) = 1, \Rightarrow$ набор выводим.

    \renewcommand{\arraystretch}{1.5} \begin{tabular}[t]{|c|c|c|}
      \hline
      \multirow{16}{*}{\(sup(ABCD)\)}
      & $\ge 0$            & $Y = ABCD$ \\\cline{2-3}

      & $\le sup(ABC) = 3$ & $Y = ABC$ \\\cline{2-3}
      & $\le sup(ABD) = 2$ & $Y = ABD$ \\\cline{2-3}
      & $\le sup(ACD) = 2$ & $Y = ACD$ \\\cline{2-3}
      & $\le sup(BCD) = 1$ & $Y = BCD$ \\\cline{2-3}
      
      & $\ge sup(ABC) + sup(ABD) - sup(AB) = 0$ & $Y = AB$ \\\cline{2-3}
      & $\ge sup(ABC) + sup(ACD) - sup(AC) = 1$ & $Y = AC$ \\\cline{2-3}
      & $\ge sup(ABD) + sup(ACD) - sup(AD) = 1$ & $Y = AD$ \\\cline{2-3}
      & $\ge sup(ABC) + sup(BCD) - sup(BC) = 1$ & $Y = BC$ \\\cline{2-3}
      & $\ge sup(ABD) + sup(BCD) - sup(BD) = 1$ & $Y = BD$ \\\cline{2-3}
      & $\ge sup(ACD) + sup(VCD) - sup(CD) = 1$ & $Y = CD$ \\\cline{2-3}
      
      & $\begin{array}{lcl} \le sup(ABC) + sup(ABD) + sup(ACD) - \\
        -sup(AB) - sup(AC) - sup(AD) + sup(A) = 1\end{array}$ & $Y = A$ \\\cline{2-3}
      
      & $\begin{array}{lcl} \le sup(ABC) + sup(ABD) + sup(BCD) - \\
        -sup(AB) - sup(BC) - sup(BD) + sup(B) = 1\end{array}$ & $Y = B$ \\\cline{2-3}
      
      & $\begin{array}{lcl} \le sup(ABC) + sup(ACD) + sup(BCD) - \\
        -sup(AC) - sup(BC) - sup(CD) + sup(C) = 1\end{array}$ & $Y = C$ \\\cline{2-3}
      
      & $\begin{array}{lcl} \le sup(ABD) + sup(ACD) + sup(BCD) - \\
        -sup(AD) - sup(BD) - sup(CD) + sup(D) = 1\end{array}$ & $Y = D$ \\\cline{2-3}

      & $\begin{array}{lcl} \ge sup(ABC) + sup(ABD) + sup(ACD) + sup(BCD) - \\
        -sup(AB) - sup(AC) - sup(AD) - sup(BC) - \\
        -sup(BD) -sup(CD) + sup(A) + sup(B) + sup(C) + \\
        +sup(D) -sup(\emptyset) = 1\end{array}$ & $Y = \emptyset$ \\
      \hline
    \end{tabular}
    \renewcommand{\arraystretch}{1}
    
    $\LB(ABCD)=\{0, 1\}, \UB(ABCD)=\{1, 2, 3\}$

    $\max(\LB(ABCD)) = \min(\UB(ABCD)) = 1, \Rightarrow$ набор выводим.
  \end{enumerate}
\end{proof}
\clearpage

\begin{task}
  Для алфавита $\{A, C, G, T\}$ посчитайте, сколько всего может быть разных
  последовательностей длины k. \\
  Даны последовательности
  \begin{center}
    \begin{tabular}{|c|c|}
      \hline
      Id & Sequence \\
      \hline
      $s_1$ & AATACAAGAAC \\
      $s_2$ & GTATGGTGAT \\
      $s_3$ & AACATGGCCAA \\
      $s_4$ & AAGCGTGGTCAA \\
      \hline
    \end{tabular}
  \end{center}
  Найдите все подпоследовательности с минимальным уровнем поддержки = 4.
\end{task}

\begin{proof}
  Возможных последовательностей длины k может быть $4^k$.

  Для поиска всех подпоследовательностей построим деревья,
  где каждое следующее дерево является продолжением предыдущего.

  \begin{center}
    \begin{forest}
      [$\emptyset$
        [A(4)] [\textbf{C(3)}] [G(4)] [T(4)]
      ]
    \end{forest}

    \begin{forest}
      [$\emptyset$
        [A(4)
          [AA(4)] [AG(4)] [AT(4)]
        ]
        [G(4)
          [GA(4)] [\textbf{GG(3)}] [\textbf{GT(2)}]
        ]
        [T(4)
          [TA(4)] [TG(4)] [\textbf{TT(2)}]
        ]
      ]
    \end{forest}

    \begin{forest}
      [$\emptyset$
        [A(4)
          [AA(4)
            [\textbf{AAA(3)}] [\textbf{AAG(3)}] [AAT(4)]
          ]
          [AG(4)
            [AGA(4)]
          ]
          [AT(4)
            [ATA(4)] [ATG(4)]
          ]
        ]
        [G(4)
          [GA(4)
            [GAA(4)]
          ]
        ]
        [T(4)
          [TA(4)
            [TAA(4)]
          ]
          [TG(4)
            [TGA(4)]
          ]
        ]
      ]
    \end{forest}

    \begin{forest}
      [$\emptyset$
        [A(4)
          [AA(4)
            [AAT(4)]
          ]
          [AG(4)
            [AGA(4)]
          ]
          [AT(4)
            [ATA(4)]
            [ATG(4)
              [ATGA(4)]
            ]
          ]
        ]
        [G(4)
          [GA(4)
            [GAA(4)]
          ]
        ]
        [T(4)
          [TA(4)
            [TAA(4)]
          ]
          [TG(4)
            [TGA(4)
              [\textbf{TGAA(3)}]
            ]
          ]
        ]
      ]
    \end{forest}
  \end{center}
\end{proof}

\end{document}
